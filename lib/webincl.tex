% Template file for Web Interface to ITRANS
% Load all languages and all fonts
% It includes all tex files, and commands needed by the web interface
% so, any change to the web form may need a change in here - when
% additional language/fontsize combinations are added, when additional
% text typesetting combinations are added (prose, lines, verse2column, etc).
% Processed by LaTeX - contains LaTeX/TeX commands only.
% April 2001, Avinash Chopde <avinash@aczoom.com> http://www.aczoom.com/
% -----------------------------------------------------------------
% Font Stuff
% Define all fonts to be of similar sizes visually - using
% dvng as the benchmark.
% Web page current has these sizes:
% small, normal, large, huge   --> dvng sizes 10, 12, 18, 36
% ------------------------------------------------------------------
% ----- telugu
\input itrnstlg   
\newfont{\tlgsmall}{tel12 at 15pt}
\newfont{\tlgnormal}{tel18}
\newfont{\tlglarge}{tel18 at 28pt}
\newfont{\tlghuge}{tel18 at 54pt}

\hyphenchar\tlgsmall=-1 % disable hyphenation using this font
\hyphenchar\tlgnormal=-1 % disable hyphenation using this font
\hyphenchar\tlglarge=-1 % disable hyphenation using this font
\hyphenchar\tlghuge=-1 % disable hyphenation using this font

% --------- kannada
\newfont{\kansmall}{kan12 at 15pt}
\newfont{\kannormal}{kan18}
\newfont{\kanlarge}{kan18 at 28pt}
\newfont{\kanhuge}{kan18 at 54pt}

\hyphenchar\kansmall=-1 % disable hyphenation using this font
\hyphenchar\kannormal=-1 % disable hyphenation using this font
\hyphenchar\kanlarge=-1 % disable hyphenation using this font
\hyphenchar\kanhuge=-1 % disable hyphenation using this font

% ------   tamil
\newfont{\tmlsmall}{wntml12 at 12pt}
\newfont{\tmlnormal}{wntml17 at 17pt}
\newfont{\tmllarge}{wntml17 at 23pt}
\newfont{\tmlhuge}{wntml17 at 48pt}

\hyphenchar\tmlsmall=-1 % disable hyphenation using this font
\hyphenchar\tmlnormal=-1 % disable hyphenation using this font
\hyphenchar\tmllarge=-1 % disable hyphenation using this font
\hyphenchar\tmlhuge=-1 % disable hyphenation using this font

% -------------- bengali
\newfont{\bngsmall}{itxbeng at 11pt}
\newfont{\bngnormal}{itxbeng at 12pt}
\newfont{\bnglarge}{itxbeng at 20pt}
\newfont{\bnghuge}{itxbeng at 40pt} 

\hyphenchar\bngsmall=-1 % disable hyphenation using this font
\hyphenchar\bngnormal=-1 % disable hyphenation using this font
\hyphenchar\bnglarge=-1 % disable hyphenation using this font
\hyphenchar\bnghuge=-1 % disable hyphenation using this font

% ------------ gujarati
\newfont{\gujsmall}{itxgujre at 12pt}
\newfont{\gujnormal}{itxgujre at 15pt}
\newfont{\gujlarge}{itxgujre at 24pt}
\newfont{\gujhuge}{itxgujre at 44pt}

\hyphenchar\gujsmall=-1 % disable hyphenation using this font
\hyphenchar\gujnormal=-1 % disable hyphenation using this font
\hyphenchar\gujlarge=-1 % disable hyphenation using this font
\hyphenchar\gujhuge=-1 % disable hyphenation using this font

% --------- Devanagari
% idevn sizes: small (9) < normal (10) < large (11) < Large (12)
%      < LARGE (14) < huge (17) < Huge (21)
% idevn.tex contains devanagari stuff, as well as definitions for ITRANS
% song book .s files.

\input idevn
\franstrue

\newfont{\dvnbsmall}{dvnb10 at 10pt}
\newfont{\dvnbnormal}{dvnb10 at 12pt}
\newfont{\dvnblarge}{dvnb10 at 18pt}
\newfont{\dvnbhuge}{dvnb10 at 40pt}

\hyphenchar\dvnbsmall=-1 % disable hyphenation using this font
\hyphenchar\dvnbnormal=-1 % disable hyphenation using this font
\hyphenchar\dvnblarge=-1 % disable hyphenation using this font
\hyphenchar\dvnbhuge=-1 % disable hyphenation using this font

% Devnac postScript font
% \newfont{\devnfsmall}{dnh}
% \newfont{\devnfnormal}{dnh at 15pt}
% \newfont{\devnflarge}{dnh at 25pt}
% \newfont{\devnfhuge}{dnh at 40pt}

% ------------ gurmukhi (punjabi)
\newfont{\grmsmall}{pun at 12pt}
\newfont{\grmnormal}{pun at 15pt}
\newfont{\grmlarge}{pun at 24pt}
\newfont{\grmhuge}{pun at 44pt}

\hyphenchar\grmsmall=-1 % disable hyphenation using this font
\hyphenchar\grmnormal=-1 % disable hyphenation using this font
\hyphenchar\grmlarge=-1 % disable hyphenation using this font
\hyphenchar\grmhuge=-1 % disable hyphenation using this font

% ------------ romancsx
% CSUtopia: putr8i
% NCS_CSX+: ncprcsxp
% Times_CSX+:  tcprcsxp
% use similar font as used by english (engsmall etc)

\newfont{\rcsxsmall}{ncprcsxp}
\newfont{\rcsxnormal}{ncprcsxp at 12pt}
\newfont{\rcsxlarge}{ncprcsxp at 18pt}
\newfont{\rcsxhuge}{ncprcsxp at 36pt}

% ------------ english
% english fonts cmr/cmbx looks bad in PDF,
% use New Century Schoolbook or Times
%  Times looks best when PDF files are viewed using Acrobat, but Acrobat
%  is very poor at displaying any other fonts (Times_CSX+ etc).
% gsview/ghostscript is very good at displaying all fonts in PDF files,
% so sticking with New Century Schoolbook for both English and Roman CSX
% ncs: pncr8r
% times: ptmr8r 

% texlive 2008: does not have pncr fonts ...
% \usepackage{newcent}
% \newfont{\ncssmall}{pncr8r at 11pt}
% \newfont{\ncsnormal}{pncr8r at 12pt}
% \newfont{\ncslarge}{pncr8r at 19pt}
% \newfont{\ncshuge}{pncr8r at 40pt}

% \usepackage{times}
% \newfont{\tmsmall}{ptmr8r at 12pt}
% \newfont{\tmnormal}{ptmr8r at 14pt}
% \newfont{\tmlarge}{ptmr8r at 21pt}
% \newfont{\tmhuge}{ptmr8r at 46pt}

\newfont{\cmrsmall}{cmr10}
\newfont{\cmrnormal}{cmr12}
\newfont{\cmrlarge}{cmr17}
\newfont{\cmrhuge}{cmr17 at 36pt}

% \newfont{\cmbxsmall}{cmbx10}
% \newfont{\cmbxnormal}{cmbx12}
% \newfont{\cmbxlarge}{cmbx17}
% \newfont{\cmbxhuge}{cmbx17 at 36pt}

% \newcommand{\engsmall}{\let\englfont=\ncssmall\englfont}
% \newcommand{\engnormal}{\let\englfont=\ncsnormal\englfont}
% \newcommand{\englarge}{\let\englfont=\ncslarge\englfont}
% \newcommand{\enghuge}{\let\englfont=\ncshuge\englfont}
\newcommand{\engsmall}{\let\englfont=\cmrsmall\englfont}
\newcommand{\engnormal}{\let\englfont=\cmrnormal\englfont}
\newcommand{\englarge}{\let\englfont=\cmrlarge\englfont}
\newcommand{\enghuge}{\let\englfont=\cmrhuge\englfont}
% ----------------------
\def\`{{\englfont ``}} % use \` for right double quote in hindi text
\def\'{{\englfont ''}} % use \' for left double quote in hindi text
\def\-{{\englfont -}} % use \- for dash in hindi text

\def\.{{\englfont .}}% use \. for period in hindi -text
\def\m+{\sBs{-0.30}{\char32}\kRn{-0.5}\sBs{0.50}{\char94}}

\newcommand{\SCOUNT}{\stepcounter{scounter}\arabic{scounter}}
\newcounter{scounter}
\newcommand{\RCOUNT}{\stepcounter{rcounter}\arabic{rcounter}}
\newcounter{rcounter}

\renewcommand{\arraystretch}{1.25}
\newcommand{\rarrow}{\mbox{---\hspace{-8pt}$>$}}

% ------------------------------------------------------------
% The key font changing commands. Current textsize is held
% in a variable, combining script name suffix with textsize
% gives the correct font to use.

\def\itrfont#1{\csname #1\itrsize\endcsname}

% define commands that will be used for every #<script>font ITRANS command
% can't use #indianfont=\itrfont{dvng} in ITRANS ( { } not supported), need
% to define #indianfont=\itrfontdvng and tml tlg etc
\def\itrfontdvnb{\itrfont{dvnb}\specialsforfrans}
\def\itrfonttml{\itrfont{tml}\specialsforfrans}
\def\itrfonttlg{\itrfont{tlg}}
\def\itrfontbng{\itrfont{bng}}
\def\itrfontgrm{\itrfont{grm}}
\def\itrfontrcsx{\itrfont{rcsx}}
\def\itrfontkan{\itrfont{kan}}
\def\itrfontguj{\itrfont{guj}}
\def\itrfonteng{\itrfont{eng}}

% define commands that will be used to set current text size - used by
% the web interface CGI script directly. Again, no { } chars allowed
% in command names/arguments.
\def\itrsizesmall{\normalbaselineskip=15pt\normalbaselines\gdef\itrsize{small}}
\def\itrsizenormal{\normalbaselineskip=18pt\normalbaselines\gdef\itrsize{normal}}
\def\itrsizelarge{\normalbaselineskip=25pt\normalbaselines\gdef\itrsize{large}}
\def\itrsizehuge{\normalbaselineskip=54pt\normalbaselines\gdef\itrsize{huge}}
% XXX - April 2001 - The baselineskip is not working, lines are
% still too close together....
% -- that is because baselines for a paragraph are computed when the next
% paragraph is begun... use baselinestretch instead?

\def\itrsize{normal}\itrfonteng % default
% ------------------------------------------------------------
% override some isongs definitions, for the web page use
\def\songfile{Online Input}
\def\year#1{} % missing from idevn.tex
% the CGI script may redefine \songfile if needed
% ------------------------------------------------------------

% web interface uses following commands to enable line formats

\newenvironment{itrprose}{}{}
\newenvironment{itrlines}{\obeylines}{}
\newenvironment{itrverse}{\obeylines\obeyspaces}{}
\newenvironment{itrversetwocol}{\obeylines\obeyspaces\begin{multicols}{2}}{\end{multicols}}
% ------------------------------------------------------------
